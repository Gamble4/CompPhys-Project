\section{Präsentation}

Ideen für die Präsentation

\begin{itemize}
    \item FEM Formulierung für CN aufschreiben?
    \item Vergleich gekühlt und ungekühlt
    \item Vergleich der drei bei gleichen Prozessen Kühlemthoden
    \item firedrake optimisation und firebar graphen
    \item Mass Lumping und Mudot - Ku = f Formulierung
    \item Robin Randbedingungen
\end{itemize}


Zu Vergleich mit den gleichen Prozessen. 
Auf jeden Fall auf die Diffusionskonstante aufpassen! 0.01 für CPU embed
da Heatpipe 1 haben
für die CPU immer die gleiche Verwenden.
D groß in Heatpipe
D kleiner in stock cooler am Rand. (als Heatpipe)

Bei Lumping, besondere Massematrix (Zeilensumme) in der CG1 Basis, das die Knoten
\begin{equation}
    \sum_j \int \phi_i \phi_j dx = \int \phi_i, ~~ \sum_j \phi_j = 1
\end{equation}
nur auf dem Punkt selbst 1 sind, an den Nachbarn sonst 0.
geht nicht mehr auf höheren Ordnungen, e.g quadratische Lagr-Interpolation