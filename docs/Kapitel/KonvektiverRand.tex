\section{Konvektiver Rand}

Es werden Robin-Randbedingungen verwendet, um einen konvektiven Rand zu simulieren, siehe \eqref{eq:KonvektiverRand}.
Wir setzen für dieses Beispiel:

Die verwendeten Größen:
\[
    \begin{array}{lcr}
        R_I & ... & \text{innerer Radius des Zylinders} \\
        R_O & ... & \text{äußerer Radius des Zylinders} \\
        T_W & ... & \text{Wassertemperatur } = 10 \\
        T_0 & ... & \text{Basistemperatur } = 40 \\
        \rho & ... & \text{Zylinderkoordinate } = \sqrt{x^2 + y^2}
    \end{array}
\]


\[
    c|_\Gamma = c(t, x), ~~ c = T_W. \text{(Wassertemperatur)}, ~~ \alpha = 1
\]

Da wir in der Differentialgleichung allerdings den Term $ \overline{D} (\nabla c) \cdot n\propto - \partial_t q_n$ haben, dreht sich in der Implementation das Vorzeichen.

Als Anfangsbedingungen:

\[
    c(0, x) = 20 \cos(\frac{\rho }{(2 \pi R_I)}) + T_0
\]
Außerdem für die isolierende Schicht ein lineares Profil der Form:
\[
    c(0, x) = T_0 - \frac{\rho - R_I}{R_O - R_I} (T_0 - T_W)
\]

Außerdem wollen wir einen Quellterm $f$:

\[
    f(t, x) = X_{[10, 20]}[\frac{c(0, x)}{5}]
\]

Dieser wirkt in einem Untervolumen.

Die zu lösende Gleichung nimmt die folgende Form an:

\begin{equation}
    \begin{aligned}
        &\int_{\Omega}  c^{n+1} \, v \, + h\overline{D}^{n+1} \nabla c^{n+1} \nabla v \, dV         
        = \int_{\Omega}  (c^n + h f^{n+1}) \, v \, dV +
        h \int_{\partial \Omega} \overline{D}^{n+1} (-c^{n+1}+10)\, v \, dS \\
        &\int_{\Omega}  c^{n+1} \, v \, + h\overline{D}^{n+1} \nabla  c^{n+1} \nabla v \, dV + h \int_{\partial \Omega}  \overline{D}^{n+1} c^{n+1} \, v \, dS 
        = \int_{\Omega}  (c^n + h f^{n+1}) \, v \, dV
        + 10h  \int_{\partial \Omega} \overline{D}^{n+1} \, v \, dS
    \end{aligned}
\end{equation}



possible sources:
    konvektion : https://www.minhnguyencae.info/weak-formulation-for-the-heat-transfer-problem/
    zeit-int in weak: https://mat1.uibk.ac.at/heiko/acme16.pdf
    erzwungene Konvektion : \url{https://www.acin.tuwien.ac.at/fileadmin/cds/lehre/mblg/Modellbildung_Kapitel_3.pdf}
    ortsabhängige Funktion, Zelle 3 : \url{https://www.firedrakeproject.org/firedrake/demos/DG_advection.py.html}