\section{Kapitel 1: Grundlagen} \label{sec:Kapitel1}

\subsection{Definitionen und Notation}

Im Folgenden werden folgende Bezeichnungen verwendet:

\[
    t \in \mathbb{R}_+, ~~~ x \in \Omega \subset \mathbb{R}^3 
\]

\begin{definition}[Temperaturfeld]
    \[
        c : \mathbb{R}_+ \times \Omega \rightarrow \mathbb{R}_+
    \]
    Im Allgemeinen von der Form $c(t, x)$.
\end{definition}

\begin{definition}[Diffusionskoeffizient]
    \[
        \overline{D} : \mathbb{R}_+ \times  \Omega \times \mathbb{R}_+ \rightarrow \Omega
    \]
    Im Allgemeinen ein Tensor zweiter Stufe von der Form $\overline{D}(t, x, c)$.
\end{definition}

Gibt die unterschiedliche Diffusion in verschiedenen Teilen des Materials an. Kann mithile 
der Tensorform Diffusionskoeffizienten, die in Ort und Durchflussrichtung variieren berücksichtigen.
e.g $\overline{D}_{xy}$ gibt den Diffusionskoeffizienten einer in y-Richtung orientierten Fläche des Volumens, für einen Strom, der in x-Richtung fließt. %Im Falle des karthesischen Koordinaten,
%also das die  Indices $ i, j \in \{x, y, z \}$, sind nimmt $\overline{D}$ Diagonalgestalt an.

Im Rahmen dieser Arbeit wird der Diffusionskoeffizient stets in Diagonalgestalt auftreten  
entweder als Skalar ( = alle Diagonaleinträge gleich sowie konstant, Material wird als lokal isotrop angenommen), oder als Diagonalmatrix 
der Form $\overline{D}_{ii}(t, x, c) = g_i(t, x, c)$. Mit $g_i$ den entsprechenden Funktionen. Querströme werden nicht berücksichtigt.

\begin{definition}[Reaktionsterm]
    \[
        f : \mathbb{R}_+ \times  \Omega \times \mathbb{R}_+ \rightarrow \mathbb{R}
    \]
    Im Allgemeinen von der Form $f(t, x, c)$. 
\end{definition}

Gibt mögliche Quellterme an, darunter können auch chemische Reaktionen fallen.


\begin{definition}[Wärmestromdichte]
    \[
        \partial_t{q}(t, x) := - \lambda \nabla c(t, x)
    \]
    $\lambda$ die Wärmeleitfähigkeit. Wird auch als Fourier'sches Gesetz bezeichnet.
\end{definition}

Im engen Zusammenhang dazu, und später für die Neumann-Randbedingungen relevant.

\begin{definition}[Wärmediffusivität]
    \[
        \overline{D} = \frac{\lambda}{\rho c_{\rho}}
    \]
    $\lambda$ die Wärmeleitfähigkeit. $\rho$ die Dichte, $\rho_c$ die 
    spezifische Wärmekapazität.
\end{definition}

Dieser Zusammenhang gilt insbesondere in Festkörpern, der obere Ausdruch ist noch allgemeiner. Die spezifische Wärmekapazität kann selbst eine Funktion der Temperatur sein, so auch die Dichte des Materials. Wir wollen hier allerdings 
annehmen, dass diese Größen konstant bleiben, da auch die Temperaturunterschiede nicht so groß sind.



\begin{definition}[charekteristisches Funktional]
    \[
        X_{[t_0, t_1]}[f(t)] := \begin{cases}
            f(t), & \text{falls } t \in [t_0, t_1] \\
            0, & \text{sonst}
        \end{cases}
    \]
\end{definition}


\subsection{Zentrale Gleichungen}

Damit können wir die zentrale Wärmeleitungsgleichung schreiben:
\begin{definition}[Wärmeleitungsgleichung]
    \begin{equation}
        \partial_t c = \nabla ( \overline{D} \,  \nabla c) + f
        \label{eq1:Wärmeleitungsgleichung}        
    \end{equation}    
\end{definition}


\subsection{Zeitintegration}

Wir verwenden als Zeitintegration die implizite Euler-Methode. Diese ist ein geeignetes Verfahren für Probleme, die nach einiger Zeit 
einen stationären Zustand einnehmen. Außerdem ist sie einfach zu Implementieren und günstig in der Berechnung, da sie nur eine Funktionsauswertung benötigt.


\begin{definition}[Implizite Euler-Methode]
    \[
        u^{n+1} = u^n + h f(t^{n+1}, u^{n+1})
    \]
    
\end{definition}

\begin{satz}
    Die implizite Euler-Methode ist A-stabil und von der Ordnung $O(h)$.
\end{satz}

\begin{proof}
    Siehe Vorlesung
\end{proof}

Das unterliegende Problem ist $\partial_t u = f(t, u)$. Obere Indices geben
die zeitlichen Schritte. $h$ ist der Zeitschritt.


Darin wird die diskretisierte Wärmeleitungsgleichung eingesetzt:

\[
    c^{n+1} = c^n + h (  \nabla ( \overline{D}^{n+1} \,  \nabla c^{n+1}) + f^{n+1} )        
\]

Diese Gleichung kommt in den Galerkin-Ansatz, um daraus eine FEM-Methode 
zu erhalten:
\begin{equation}    
    \begin{aligned}        
        \int_{\Omega}  c^{n+1} \, v \, dV
        &= \int_{\Omega}  c^n \, v \, dV + h \int_{\Omega} (  \nabla ( \overline{D}^{n+1} \,  \nabla c^{n+1}) + f^{n+1} ) \, v \, dV \\
        &= \int_{\Omega}  (c^n + h f^{n+1}) \, v \, dV +
        h \int_{\partial \Omega} \overline{D}^{n+1} (\nabla c^{n+1} \cdot n) \, v \, dS 
        - h \int_{ \Omega} \overline{D}^{n+1} \nabla c^{n+1} \nabla v \, dV 
    \end{aligned}
    \label{eq2:FEM}    
\end{equation}





\subsection{Zu den Randbedingungen}

Hier wird der Rand des Gebietes $\Omega$ mit $\Gamma$ bezeichnet.

\[
    \partial \Omega := \Gamma
\]

Besondere Bedeutung besitzt folgender Ausdruck aus Gleichung \eqref{eq2:FEM}:

\[
    \overline{D} (\nabla c \cdot n)
\]  

er hängt mit der Wärmestromdichte am Rand zusammen.

\begin{equation}
    \partial_t q|_{\Gamma} = - \lambda \nabla c = - \rho c_\rho \overline{D} (\nabla c \cdot n)
    \label{eq3:Randterm}
\end{equation}

\subsection{Robin-Randbedingungen und konvektiver Rand}

Sind eine Mischung aus Dirichlet- sowie Neumann Randbedingungen 
und haben die allgemeine Form:

\[
    a c(t, x) + b \frac{\partial c(t, x)}{\partial n} = g(t, x), ~~~ x \in \Gamma
\]

Wobei $c, g$ Funktionen sind und $a, b$ Konstanten.

Sie werden verwendet um einen konvektiven Rand zu simulieren. 
Dieser lässt sich im einfachen Falle von laminaren, stationären Strömen schreiben als:

\begin{equation*}
    \begin{aligned}
        b \frac{\partial c}{\partial n} &= -\partial_t q_n 
        = g - ac \\
        \text{setze: } b &= \lambda,~~ g = a \, c|_\Gamma, ~~ \alpha := a/\lambda   \\
        \partial_t q_n &= \alpha (c - c|_\Gamma)
    \end{aligned}
\end{equation*}

Die Konstante kann für verschiedene Abschnitte des Randes $\Gamma_a \subset \Gamma$ unterschiedlich festgelegt werden, damit:

\begin{equation}
    \partial_t q_n = \alpha(x) (c(t, x) - c(t, x)|_\Gamma)
    \label{eq:KonvektiverRand}
\end{equation}



\subsection{Problemstellung}

Wir wollen die Wärmeleitung in Feststoffen untersuchen. Anfangs ein einfaches Beispiel eines Rohres 
und anschließend eventuell ein einfaches elektronisches Bauteil.

Dazu muss zu jedem Zeitschritt die rechte Seite der Gleichung \eqref{eq1:Wärmeleitungsgleichung} 
gelöst werden. Dafür wird Firedrake verwendet. Die Zeitintegration kann mittels eines einfachen Integrators, zum Beispiel dem Euler-Verfahren gelöst werden, oder auch mit PETSC. 


