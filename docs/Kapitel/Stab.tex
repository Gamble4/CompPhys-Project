\section{Bsp 1: Stab}

\subsection{Modell}
Erstes Beispiel, ein Stab, entland dessen sich die Wärme ausbreitet. Mit einfacher
Wärmeleitungsgleichung ohne Quellen und konstantem Diffusionskoeffizienten.

\[
    \overline{D}_{ii}^n  = 1 ~~ \forall i,n\, ,  ~~
    f = 0
\]

Damit reduziert sich die zu lösende Gleichung \eqref{eq1:Wärmeleitungsgleichung}:

\begin{equation}
    \partial_t c =  \Delta c
\end{equation}


\subsection{Randbedingungen}

Als Anfangstemperaturverteilung soll gelten:

\[
    c(0, x) = 0
\]

Als Randbedingungen des Rohres wollen wir keinen Temperaturfluss über dessen Mantel und einen konstanten Fluss über eine einzelne Seite, wir setzen an einer der Randflächen:

\[
    \overline{D} \nabla c \cdot n = -1
\]

Das negative Vorzeichen bewirkt, dass es sich um einen Temperaturfluss 
aus dem Volumen handelt. Ein positives Vorzeichen würde bedeuten, dass 
auf dem Rand der Gradient $\nabla c > 0$, damit ist außerhalb des Volumens die 
Temperatur größer als innerhalb. Da die Wärme entgegen dem 
Temperaturgradienten fließt, steigt die Temperatur im Inneren des Volumens.

Die einzigen Quellen soll ein konstanter Temperaturwert an der anderen Randfläche des Rohres sein.
Damit vereinfacht sich die allgemeine Formel \eqref{eq2:FEM} zu:

\[
\int_{\Omega}  c^{n+1} \, v  + h \nabla c^{n+1} \nabla v \, dV           
        = \int_{\Omega}  c^n \, v \, dV +
        h\, \int_{\partial \Omega} \, v \, dS 
\]

